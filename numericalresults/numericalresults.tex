\documentclass[12pt]{article}
\linespread{1}

\usepackage{amsmath}
\usepackage{amsthm}
\usepackage{amssymb}
\usepackage{graphicx}
\usepackage{listings}
\usepackage{tabularx}
\usepackage{framed}
\usepackage{verbatim}
\usepackage{float}
\usepackage{hyperref}
\usepackage{color}
\usepackage[toc,page]{appendix}
\usepackage{mathrsfs,mathtools,esint}

\hfuzz=2000pt 

%\usepackage{fullpage}

\hoffset  0 true in
\voffset   0 true pt
\textheight  9 true in
\textwidth  6 true in 
\oddsidemargin 0 true in
\topmargin 0 true in
\headsep  0 true in
\headheight  0 true in
\footskip   .5 true in

\newtheorem{Riemann}{Riemann Theorem}
\newcommand{\ds}{\displaystyle}
\renewcommand{\arraystretch}{1.5}


\title{Numerical Results: Multiple time stepping algorithms for explicit one-step exponential integrators} 
\author{} 
\date{\today } 

\begin{document}
\maketitle
\noindent
\section{Problem definition}
\emph{Goal of paper:} \\
Construct efficient (MTS) algorithms based on classes of explicit one-step exponential integrators.\\

\emph{What's desirable about these methods?}\\
\begin{itemize}
\item Application to exponential Rosenbrock methods
\item Do not require computation of starting values
\item Easy implementation of adaptive time step control
\item Avoid matrix functions
\item Order of accuracy up to four to be demonstrated
\end{itemize}

\noindent
\emph{Problem we're solving:}
\begin{equation} 
u'(t) = F(t,u(t)) = Au(t) + g(t,u(t)),  u(t_0) = u_0 \tag{1.1} \label{eq: 1.1}
\end{equation}
on the interval $t_0 \leq t \leq T$, where the vector field $F(t,u(t))$ can be decomposed into a linear stiff part $Au(t)$ and nonlinear nonstiff part $g(t,u(t))$. We consider the case where the stiff part is cheap to compute and the nonstiff part is expensive.

\subsection{MTS Algorithm for ETD Methods}
\begin{equation}
\hat{p}_{n,i}(\tau) = \sum_{j=1}^{i-1} \Bigg(\sum_{k=1}^{l_{ij}} \frac{\alpha_{ij}^{(k)}}{c_i^kh^{k-1}(k-1)!}\tau^{k-1}\Bigg)g(t_n + c_jh,\hat{U}_{n,j}),
\tag{3.19a} \label{eq: 3.19a}
\end{equation}
\begin{equation}
\hat{q}_{n,s}(\tau) = \sum_{i=1}^{s} \Bigg(\sum_{k=1}^{m_i} \frac{\beta_{i}^{(k)}}{h^{k-1}(k-1)!}\tau^{k-1}\Bigg)g(t_n + c_ih,\hat{U}_{n,i}),
\tag{3.19b} \label{eq: 3.19b}
\end{equation}
To get $\hat{u}_{n+1}$ solve: 
\begin{equation}
y'_{n,i}(\tau) = Ay_{n,i}(\tau) + \hat{p}_{n,i}(\tau), \hspace{0.3cm} y_{n,i}(0) = \hat{u}_n \hspace{0.3cm} (2\leq i \leq s)
\tag{3.20} \label{eq: 3.20}
\end{equation}
\subsubsection{Strategy for computing $\hat{U}_{n,i}$}
\begin{enumerate}
\item Set $\hat{U}_{n,1} = \hat{u}_n$.
\item $\hat{U}_{n,1}$ is now known. Evaluate $\hat{p}_{n,2}(\tau)$ from (\ref{eq: 3.19a}) and solve (\ref{eq: 3.20}) with $i=2$ to obtain $\hat{U}_{n,2} \approx \hat{y}_{n,2}(c_2h)$.
\item $\hat{U}_{n,1}, \hat{U}_{n,2} $ is now known. Evaluate $\hat{p}_{n,3}(\tau)$ from (\ref{eq: 3.19a}) and solve (\ref{eq: 3.20}) with $i=3$ to obtain $\hat{U}_{n,3} \approx \hat{y}_{n,3}(c_3h)$.\\
$\vdots$
\item $\hat{U}_{n,1},\cdots \hat{U}_{n,s-1} $ is now known. Evaluate $\hat{p}_{n,s}(\tau)$ from (\ref{eq: 3.19a}) and solve (\ref{eq: 3.20}) with $i=s$ to obtain $\hat{U}_{n,s} \approx \hat{y}_{n,s}(c_sh)$.
\end{enumerate}

\begin{equation}
y'_{n}(\tau) = Ay_{n}(\tau) + \hat{q}_{n}(\tau), \hspace{0.3cm} y_{n}(0) = \hat{u}_n 
\tag{3.21} \label{eq: 3.21}
\end{equation}
\section{Numerical Tests} 
\subsection{A third order method \texttt{expRK32}} 
Consider the three stage ETD method of stiff order three with the following Butcher tableau.
\begin{center}
\begin{table}[h!]
\begin{tabular}{r|c c c}
$0$ &  &  &  \\
$c_2$ & $c_2\varphi_{1,2}$ &  \\
$\frac{2}{3}$ & $\frac{2}{3}\varphi_{1,3} - \frac{4}{9c_2}\varphi_{2,3}$ & $\frac{4}{9c_2}\varphi_{2,3}$\\
\hline 
 & $\varphi_1 - \frac{3}{2}\varphi_2 $ & $0$ & $\frac{3}{2}\varphi_2$\\
 & $\varphi_1$ & $\frac{3}{3c_2 - 2}\varphi_2$ & $\frac{3}{2 - 3c_2}\varphi_2$ 
\end{tabular}
\end{table}
\end{center}

From the Butcher tableau, \\
$l_{21} = q_1 = 1$,\\
$l_{31} = l_{32} = m_1 = m_3 = q_2 = q_3 = 2$,\\
$\alpha_{21}^{(1)} = c_2 , \alpha_{31}^{(1)} = \frac{2}{3}, \alpha_{32}^{(1)} = 0$,\\
$\beta_1^{(1)} = 1, \beta_1^{(2)} = -\frac{3}{2}, \beta_2^{(k)} = 0 \hspace{.2cm} \forall k = 1,...,m_2$,\\
$\beta_3^{(1)} = 0, \beta_3^{(2)} = \frac{3}{2}$,\\ 
$\bar{\beta}_2^{(1)} = \bar{\beta}_3^{(1)} = 0$,\\
$\bar{\beta}_2^{(2)} = -\bar{\beta}_3^{(2)} = \frac{3}{3c_2 - 2} (c_2 \neq \frac{2}{3})$.\\

We then have:
\begin{equation}
\hat{p}_{n,2}(\tau) = g(t_n,\hat{u}_n), \tag{3.30a} \label{eq: 3.30a}
\end{equation}
\begin{equation}
\hat{p}_{n,3}(\tau) = \frac{1}{c_2h}g(t_n + c_2h,\hat{U}_{n,2})\tau + (1- \frac{\tau}{c_2h})\hat{p}_{n,2}(\tau), \tag{3.30b} \label{eq: 3.30b}
\end{equation}
 \begin{equation}
 \hat{q}_{n,3}(\tau) = \frac{3}{2h}g(t_n + \frac{2}{3}h, \hat{U}_{n,3}) \tau + (1 - \frac{3\tau}{2h})\hat{p}_{n,2}(\tau), \tag{3.30c} \label{eq: 3.30c}
 \end{equation}
 \begin{equation}
 \hat{\bar{q}}_{n,3}(\tau) = \frac{3}{(3c_2-2)h}(g(t_n + c_2h,\hat{U}_{n,2}) - g(t_n + \frac{2}{3}h, \hat{U}_{n,3}) )\tau + \hat{p}_{n,2}(\tau). \tag{3.30c} \label{eq: 3.30d}
 \end{equation}
 
%_____________________________________________________________________________%
%________________________Brusselator Problem__________________________________%
%_____________________________________________________________________________%
\subsubsection{Example: Brusselator Problem}
We consider the Brusselator problem represented by  \emph{(Why?)}
\begin{equation}
\mathbf{\frac{dy}{dt}} = \begin{bmatrix}
a - (y_3 + 1)y_1 + y_1^2y_2\\ y_3y_1 - y_1^2y_2\\ \frac{b-y_3}{\epsilon} - y_1y_3
\end{bmatrix}
\end{equation}
This equation in the form of (\ref{eq: 1.1}) becomes
\begin{equation}
\mathbf{\frac{dy}{dt}} = \begin{bmatrix}
0 & 0 & 0 \\ 0 & 0 & 0 \\ \frac{-1}{\epsilon} & 0 & 0
\end{bmatrix} \begin{bmatrix}
y_1 \\ y_2 \\ y_3
\end{bmatrix}  + \begin{bmatrix}
a - (y_3 + 1)y_1 + y_1^2y_2\\ y_3y_1 - y_1^2y_2\\ \frac{b}{\epsilon} - y_1y_3
\end{bmatrix}
\end{equation}
where $y_1(0) = 1.2, y_2(0) = 3.1,$ and $y_3(0) = 3$, with parameters $a=1, b=3.5,$ and $\epsilon =0.01$.  We evaluate over the time interval $[0,10]$. We investigate the performance of the algorithm when ODE solvers of different orders are used for the inner steps. For each ODE solver, the macro time steps are given by the $h-$values and $h/m$ determines the micro time steps.\\

%______________________________________________________________________________%
%_____________________Heun Method______________________________________________%
%We consider \texttt{Heun-Euler-ERK} which is a two stage, second order method. 
%\begin{table}[h!]
%\centering
%\caption{\texttt{Heun:} $m = 5$}
%\begin{tabular}{|r |c |c |c |c|}
%\hline
%$h$ & Max. Abs. Value & Rate & RMS Value & Rate\\
%\hline
% 0.01 & 4.60815e-03  &   & 5.41703e-04   & \\
%\hline
%0.005 & 8.33194e-04   & 2.46746e+00  & 9.80561e-05  & 2.46582e+00 \\
%\hline
%0.0025 & 1.67974e-04  & 2.31042e+00  & 1.97735e-05   & 2.31004e+00 \\
%\hline
%0.00125 & 3.69695e-05 & 2.18383e+00  & 4.35232e-06 & 2.18371e+00 \\
%\hline
%\end{tabular}
%\end{table}
%
%\begin{table}[h!]
%\centering
%\caption{\texttt{Heun:} $m = 25$}
%\begin{tabular}{|r |c |c |c |c|}
%\hline
%$h$ & Max. Abs. Value & Rate & RMS Value & Rate\\
%\hline
% 0.01 &2.39217e-03   &   & 2.81329e-04  & \\
%\hline
%0.005 & 3.13373e-04  & 2.93237e+00  & 3.68804e-05  & 2.93133e+00\\
%\hline
%0.0025 & 4.22839e-05  & 2.88970e+00  & 4.97713e-06  & 2.88947e+00 \\
%\hline
%0.00125 & 6.06845e-06  & 2.80071e+00  &7.14338e-07   & 2.80064e+00  \\
%\hline
%\end{tabular}
%\end{table}
%
%\begin{table}[h!]
%\centering
%\caption{\texttt{Heun:} $m = 50$}
%\begin{tabular}{|r |c |c |c |c|}
%\hline
%$h$ & Max. Abs. Value & Rate & RMS Value & Rate\\
%\hline
% 0.01 & 2.31741e-03  &   &2.72541e-04   & \\
%\hline
%0.005 & 2.94885e-04  & 2.97429e+00  &3.47047e-05   & 2.97327e+00 \\
%\hline
%0.0025 & 3.76990e-05  & 2.96755e+00  &4.43748e-06   &2.96732e+00 \\
%\hline
%0.00125 &4.92717e-06  & 2.93569e+00  & 5.79999e-07 & 2.93562e+00 \\
%\hline
%\end{tabular}
%\end{table}
%
%\begin{table}[h!]
%\centering
%\caption{\texttt{Heun:} $m = 100$}
%\begin{tabular}{|r |c |c |c |c|}
%\hline
%$h$ & Max. Abs. Value & Rate & RMS Value & Rate\\
%\hline
% 0.01 & 2.29593e-03  &   & 2.70015e-04  & \\
%\hline
%0.005 &2.89527e-04   & 2.98730e+00   &3.40743e-05   &2.98629e+00 \\
%\hline
%0.0025 &3.63651e-05   & 2.99307e+00  &4.28049e-06   &2.99283e+00 \\
%\hline
%0.00125 &4.59450e-06  & 2.98458e+00  &5.40843e-07   &2.98449e+00  \\
%\hline
%\end{tabular}
%\end{table}
%\ \\
%\newpage
%%______________________________________________________________________________%
%%_________________________ERK-2-2______________________________________________%
%We also consider \texttt{ERK-2-2} which is a two stage, third order method.
%
%\begin{table}[h!]
%\centering
%\caption{\texttt{ERK-2-2:} $m = 5$}
%\begin{tabular}{|r |c |c |c |c|}
%\hline
%$h$ & Max. Abs. Value & Rate & RMS Value & Rate\\
%\hline
% 0.01 &4.60815e-03   &   & 5.41703e-04  & \\
%\hline
%0.005 &8.33194e-04   & 2.46746e+00  & 9.80561e-05  & 2.46582e+00\\
%\hline
%0.0025 &1.67974e-04   & 2.31042e+00  &1.97735e-05   &2.31004e+00 \\
%\hline
%0.00125 &3.69695e-05  & 2.18383e+00  &4.35232e-06   & 2.18371e+00 \\
%\hline
%\end{tabular}
%\end{table}
%
%\begin{table}[h!]
%\centering
%\caption{\texttt{ERK-2-2:} $m = 25$}
%\begin{tabular}{|r |c |c |c |c|}
%\hline
%$h$ & Max. Abs. Value & Rate & RMS Value & Rate\\
%\hline
% 0.01 & 2.39217e-03  &   &2.81329e-04   & \\
%\hline
%0.005 & 3.13373e-04  & 2.93237e+00  & 3.68804e-05  & 2.93133e+00\\
%\hline
%0.0025 &4.22839e-05   & 2.88970e+00 & 4.97713e-06  & 2.88947e+00\\
%\hline
%0.00125 & 6.06845e-06 & 2.80071e+00  &7.14338e-07   &2.80064e+00  \\
%\hline
%\end{tabular}
%\end{table}
%
%\begin{table}[h!]
%\centering
%\caption{\texttt{ERK-2-2:} $m = 50$}
%\begin{tabular}{|r |c |c |c |c|}
%\hline
%$h$ & Max. Abs. Value & Rate & RMS Value & Rate\\
%\hline
% 0.01 & 2.31741e-03  &   & 2.72541e-04  & \\
%\hline
%0.005 & 2.94885e-04  &2.97429e+00   & 3.47047e-05  & 2.97327e+00\\
%\hline
%0.0025 &3.76990e-05   & 2.96755e+00  & 4.43748e-06  &2.96732e+00 \\
%\hline
%0.00125 &4.92717e-06  & 2.93569e+00  &5.79999e-07   &2.93562e+00  \\
%\hline
%\end{tabular}
%\end{table}
%
%\begin{table}[h!]
%\centering
%\caption{\texttt{ERK-2-2:} $m = 100$}
%\begin{tabular}{|r |c |c |c |c|}
%\hline
%$h$ & Max. Abs. Value & Rate & RMS Value & Rate\\
%\hline
% 0.01 &2.29593e-03   &   & 2.70015e-04  & \\
%\hline
%0.005 & 2.89527e-04  & 2.98730e+00  & 3.40743e-05  & 2.98629e+00\\
%\hline
%0.0025 & 3.63651e-05  & 2.99307e+00  &4.28049e-06   &2.99283e+00 \\
%\hline
%0.00125 &4.59450e-06  &2.98458e+00   &5.40843e-07   & 2.98449e+00 \\
%\hline
%\end{tabular}
%\end{table}

%\ \\
%\newpage
%%____________________________________________________________________________%
%%___________________________ERK-4-4__________________________________________%
%Lastly, we consider \texttt{ERK-4-4} which is a four stage, fourth order method
%\begin{table}[h!]
%\centering
%\caption{\texttt{ERK-4-4:} $m = 5$}
%\begin{tabular}{|r |c |c |c |c|}
%\hline
%$h$ & Max. Abs. Value & Rate & RMS Value & Rate\\
%\hline
% 0.01 & 2.29208e-03  &   &2.69563e-04   & \\
%\hline
%0.005 &2.87941e-04   & 2.99281e+00  & 3.38876e-05  & 2.99180e+00\\
%\hline
%0.0025 & 3.59331e-05  & 3.00239e+00  & 4.22964e-06  & 3.00215e+00\\
%\hline
%0.00125 & 4.48444e-06 & 3.00231e+00  &5.27889e-07   &3.00231e+00  \\
%\hline
%\end{tabular}
%\end{table}
%
%\begin{table}[h!]
%\centering
%\caption{\texttt{ERK-4-4:} $m = 25$}
%\begin{tabular}{|r |c |c |c |c|}
%\hline
%$h$ & Max. Abs. Value & Rate & RMS Value & Rate\\
%\hline
% 0.01 &2.28884e-03   &   &2.69182e-04   & \\
%\hline
%0.005 & 2.87751e-04  & 2.99173e+00  &3.38652e-05   & 2.99071e+00\\
%\hline
%0.0025 &3.59216e-05   &3.00190e+00   & 4.22829e-06  &3.00166e+00 \\
%\hline
%0.00125 &4.48374e-06  & 3.00208e+00  &5.27807e-07   &3.00199e+00  \\
%\hline
%\end{tabular}
%\end{table}
%
%\begin{table}[h!]
%\centering
%\caption{\texttt{ERK-4-4:} $m = 50$}
%\begin{tabular}{|r |c |c |c |c|}
%\hline
%$h$ & Max. Abs. Value & Rate & RMS Value & Rate\\
%\hline
% 0.01 & 2.28883e-03  &   & 2.69182e-04  & \\
%\hline
%0.005 & 2.87750e-04  & 2.99172e+00  &3.38652e-05   & 2.99071e+00\\
%\hline
%0.0025 & 3.59216e-05  & 3.00189e+00  & 4.22829e-06  &3.00166e+00 \\
%\hline
%0.00125 & 4.48373e-06 &3.00208e+00   &5.27806e-07   &3.00199e+00  \\
%\hline
%\end{tabular}
%\end{table}
%
%\begin{table}[h!]
%\centering
%\caption{\texttt{ERK-4-4:} $m = 100$}
%\begin{tabular}{|r |c |c |c |c|}
%\hline
%$h$ & Max. Abs. Value & Rate & RMS Value & Rate\\
%\hline
% 0.01 & 2.28883e-03  &   &2.69182e-04   & \\
%\hline
%0.005 & 2.87750e-04  &2.99172e+00   & 3.38652e-05  & 2.99071e+00\\
%\hline
%0.025 &3.59216e-05   &3.00189e+00   & 4.22829e-06  & 3.00166e+00\\
%\hline
%0.0125 &4.48372e-06  &3.00208e+00   &5.27805e-07   &  3.00200e+00\\
%\hline
%\end{tabular}
%\end{table}
%\ \\
%\subsubsection{Notes}
%\begin{itemize}
%\item Comment on what happens when we use different methods of different orders
%\item Comment on what happens when the micro time steps are reduced
%\item Are plots or tables the ideal way of presenting this information?
%\item Is it necessary to keep both the maximum absolute error and the root mean square error? Their behavior is very similar and no new information is gained by looking at both.
%\end{itemize}
%\newpage
%________________________________________________________________________________%
%_________________________Another 3rd order method_______________________________%
\subsection{Third order method \texttt{expRK32s3}}
\begin{center}
\begin{tabular}{r|c c c}
$c_1 =0$ & & & \\
$c_2 \neq \frac{2}{3}$ & $c_2\varphi_{1,2}$ & & \\
$c_3 = \frac{2}{3}$& $\frac{2}{3}\varphi_{1,3} - \frac{4}{9c_2}\varphi_{2,3}$ & $\frac{4}{9c_2}\varphi_{2,3}$ & \\
\hline
& $b_1 = \varphi_1 - b_2-b_3$ & $b_2 = \frac{-2}{3c_2(c_2 - \frac{2}{3})}\varphi_2 + \frac{2}{c_2(c_2 - \frac{2}{3})}\varphi_3$ & $b_3 = \frac{c_2}{\frac{2}{3}(c_2 - \frac{2}{3})}\varphi_2 -\frac{2}{\frac{2}{3}(c_2 - \frac{2}{3})}\varphi_3$ \\
& $\bar{b}_1 = \varphi_1 - \bar{b}_2 - \bar{b}_3$ & $\bar{b}_2 = \frac{-2}{3c_2(c_2 - \frac{2}{3})}\varphi_2$ & $\bar{b}_3 = \frac{c_2}{\frac{2}{3}(c_2 - \frac{2}{3})}\varphi_2$ 
\end{tabular}
\end{center}
Note : $\varphi_{i,j} = \varphi_i(c_jhA)$  ,  $b_i = b_i(hA)$.\\
\underline{\textbf{Polynomials}}\\
Let \begin{equation} D_{n,i} = g(t_n + c_ih,\hat{U}_{n,i}) - g(t_n,\hat{u}_n).
\end{equation}
\begin{equation}
\hat{p}_{n,2}(\tau) = g(t_n,\hat{u}_n)
\end{equation}

\begin{equation}
\hat{p}_{n,3}(\tau) = g(t_n,\hat{u}_n) + \frac{4}{9c_2} \frac{\tau}{c_3^2h}D_{n,2}
\end{equation}

\begin{equation}
\hat{q}_{n,3}(\tau) = g(t_n,\hat{u}_n) + \frac{\tau}{h} \Bigg( \frac{-2}{3c_2(c_2 - \frac{2}{3})}D_{n,2} + \frac{c_2}{\frac{2}{3}(c_2 - \frac{2}{3})}D_{n,3} \Bigg) + \frac{\tau^2}{2h^2}\Bigg(\frac{2}{c_2(c_2-\frac{2}{3})}D_{n,2} - \frac{2}{\frac{2}{3}(c_2 - \frac{2}{3})}D_{n,3} \Bigg)
\end{equation}

\begin{equation}
\hat{\bar{q}}_{n,3}(\tau) = g(t_n,\hat{u}_n) + \frac{\tau}{h} \Bigg(\frac{-2}{3c_2(c_2 - \frac{2}{3})}D_{n,2} + \frac{c_2}{\frac{2}{3}(c_2 - \frac{2}{3})}D_{n,3}\Bigg)
\end{equation}

%_________________________4th order method_______________________________________%
\subsection{Fourth order method \texttt{expRK43s6}}
\begin{center}
\begin{table}[h!]
\begin{tabularx}{19cm}{r|c| X| X| X |X}
$0$ &  &  & & &\\
$c_2$ &  & & & &\\
$c_3$ & $a_{32} = \frac{c_3^2}{c_2}\varphi_{2,3}$ &  & & &\\
$c_4$ &$a_{42} = \frac{c_4^2}{c_2}\varphi_{2,4}$& $a_{4,3} = 0$& & &\\
$c_5$ & $a_{52} = 0$ & $a_{53} = \frac{-c_4c_5^2}{c_3(c_3-c_4)}\varphi_{2,5} + \frac{2c_5^3}{c_3(c_3-c_4)}\varphi_{3,5}$ & $a_{54} = \frac{c_3c_5^2}{c_4(c_3-c_4)}\varphi_{2,5} - \frac{2c_5^3}{c_4(c_3-c_4)}\varphi_{3,5}$ & & \\
$c_6$ & $a_{62} = 0$ & $a_{63} = \frac{-c_4c_6^2}{c_3(c_3-c_4)}\varphi_{2,6} + \frac{2c_6^3}{c_3(c_3-c_4)}\varphi_{3,6}$ & $a_{64} = \frac{c_3c_6^2}{c_4(c_3-c_4)}\varphi_{2,6} + \frac{2c_6^3}{c_4(c_3-c_4)}\varphi_{3,6}$ & $a_{65} =0$ & \\
\hline 
 & $b_2 = 0$ & $b_3 = 0$ & $b_4 = 0$& $b_5 = \frac{-c_6}{c_5(c_5-c_6)}\varphi_2 + \frac{2}{c_5(c_5-c_6)}\varphi_3$ & $b_6 =\frac{c_5}{c_6(c_5-c_6)}\varphi_2 - \frac{2}{c_6(c_5-c_6)}\varphi_3$\\
 & $\bar{b}_2 = 0$ & $\bar{b}_3 = \frac{-c_4}{c_3(c_3-c_4)}\varphi_2 + \frac{2}{c_3(c_3-c_4)}\varphi_3$ & $\bar{b}_4 = \frac{c_3}{c_4(c_3-c_4)}\varphi_2 - \frac{2}{c_4(c_3-c_4)}\varphi_3$& $\bar{b}_5 = 0$ & $\bar{b}_6 = 0$ 
\end{tabularx}
\end{table}
\end{center}
Conditions to be satisfied: $c_3 \neq c_4, c_5\neq c_6, c_6 \neq \frac{2}{3}, c_5 = \frac{4c_6 -3}{6c_6 -4}.$\\
Two sets of c values: 
\begin{itemize}
\item $c_2 = \frac{1}{2} = c_3 = c_5 ; c_4 = \frac{1}{3} ; c_6 = 1.$
\item $c_2 = c_3 = \frac{1}{2}; c_4 = c_6 = \frac{1}{3} ; c_5 = \frac{5}{6}.$
\end{itemize}

\underline{\textbf{Polynomials}}
\begin{equation}
\hat{p}_{n,2}(\tau) = g(t_n,\hat{u}_n)
\end{equation}

\begin{equation}
\hat{p}_{n,3}(\tau) = g(t_n,\hat{u}_n) + \frac{\tau}{c_2h}D_{n,2}
\end{equation}

\begin{equation}
\hat{p}_{n,4}(\tau) = g(t_n,\hat{u}_n) + \frac{\tau}{c_2h}D_{n,2}
\end{equation}

\begin{equation}
\hat{p}_{n,5}(\tau) = g(t_n,\hat{u}_n) + \frac{\tau}{h}\Bigg(\frac{-c_4}{c_3(c_3 - c_4)}D_{n,3} + \frac{c_3}{c_4(c_3 - c_4)}D_{n,4} \Bigg) + \frac{\tau^2}{2h^2}\Bigg(\frac{2}{c_3(c_3-c_4)}D_{n,3} - \frac{2}{c_4(c_3 - c_4)}D_{n,4} \Bigg)
\end{equation}

\begin{equation}
\hat{p}_{n,6}(\tau) = g(t_n,\hat{u}_n) + \frac{\tau}{h}\Bigg(\frac{-c_4}{c_3(c_3 - c_4)}D_{n,3} + \frac{c_3}{c_4(c_3 - c_4)}D_{n,4} \Bigg) + \frac{\tau^2}{2h^2}\Bigg(\frac{2}{c_3(c_3-c_4)}D_{n,3} - \frac{2}{c_4(c_3 - c_4)}D_{n,4} \Bigg)
\end{equation}

\begin{equation}
\hat{q}_{n,6}(\tau) = g(t_n,\hat{u}_n) + \frac{\tau}{h}\Bigg(\frac{-c_6}{c_5(c_5 - c_6)}D_{n,5} + \frac{c_5}{c_6(c_5 - c_6)}D_{n,6} \Bigg) + \frac{\tau^2}{2h^2}\Bigg(\frac{2}{c_5(c_5-c_6)}D_{n,5} - \frac{2}{c_6(c_5 - c_6)}D_{n,6} \Bigg)
\end{equation}

\begin{equation}
\hat{\bar{q}}_{n,6}(\tau) = g(t_n,\hat{u}_n) + \frac{\tau}{h}\Bigg(\frac{-c_4}{c_3(c_3 - c_4)}D_{n,3} + \frac{c_3}{c_4(c_3 - c_4)}D_{n,4} \Bigg) + \frac{\tau^2}{2h^2}\Bigg(\frac{2}{c_3(c_3-c_4)}D_{n,3} - \frac{2}{c_4(c_3 - c_4)}D_{n,4} \Bigg)
\end{equation}
\end{document}
